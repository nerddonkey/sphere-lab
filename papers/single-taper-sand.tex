% !TEX spellcheck = en_US
%% single column for submission and review
%\documentclass[journal, onecolumn, 12pt, a4paper, draftcls]{IEEEtran}

%% double column to emulate final version (for submission and review)
\documentclass[10pt, twocolumn, twoside]{IEEEtran}

%% double column for conference
%\documentclass[conference]{IEEEtran}

\IEEEoverridecommandlockouts
\bibliographystyle{IEEEtran}

\usepackage[cmex10]{amsmath} % prevents amsmath from using a Type 3 font for math within footnotes
\interdisplaylinepenalty=2500 % (re)allows page breaks within aligned equations
\usepackage{amssymb}
\usepackage{bm}
\usepackage{mathtools}
\usepackage{cite}
\usepackage{xcolor}
\usepackage{enumerate}
\usepackage{booktabs} % nice table objects
\usepackage{multirow}
\usepackage{lipsum}
\usepackage{graphicx}
\usepackage{breqn}
\usepackage{subcaption}

\usepackage{mathrsfs}
\usepackage{mathtools}
\usepackage{etoolbox}
\usepackage{hyperref}
%\usepackage{bibunits}
\usepackage{chngcntr}

\counterwithin*{equation}{section}
\graphicspath{{figs/}{pdf/}{jpg/}{../pdf/}}

%commands
%\renewcommand{\phi}{\varphi}
\newcommand{\untsph}{\mathbb{S}^{2}} % unit sphere
\newcommand{\unit}[1]{\widehat{\bm{#1}}}
\DeclarePairedDelimiterX\abs[1]{\lvert}{\rvert}{#1}
\DeclarePairedDelimiterX\parn[1]{(}{)}{#1}
\DeclarePairedDelimiterX\set[1]{\lbrace}{\rbrace}{#1}
\DeclarePairedDelimiterX\innerp[2]{\langle}{\rangle}{#1,#2}
\DeclarePairedDelimiterX\norm[1]{\lVert}{\rVert}{#1}
\DeclarePairedDelimiterX\brac[1]{[}{]}{#1}
\DeclarePairedDelimiterX\coeff[1]{(}{)}{#1}
\DeclareMathOperator{\expectop}{\mathbb{E}} % \expect[\Big]{ f }
\newcommand{\expect}[2][]{\expectop\set[#1]{#2}}

\newcommand{\figref}[1]{Fig.\,\ref{#1}}
\newcommand{\tabref}[1]{Table~\ref{#1}}
\newcommand{\reals}{\mathbb{R}} % real numbers
\newcommand{\cmplx}{\mathbb{C}} % complex numbers
\newcommand{\dfn}{\triangleq}
\newcommand{\conj}[1]{\overline{#1}} % conjugate
\newcommand{\tendsto}{\rightarrow}

\newtheorem{theorem}{Theorem}%[section]
\renewcommand{\IEEEQED}{\IEEEQEDopen}

%\defaultbibliographystyle{IEEEtran}
%\defaultbibliography{IEEEabrv,book-collection,sphtriang}

\begin{document}

\title{Single-Taper Window Design on the Sphere}

\ifCLASSOPTIONconference
	\author{
		\IEEEauthorblockN{Rodney~A.~Kennedy}
			\thanks{This research was supported under the Australian Research Council's Discovery Projects
				funding scheme (Project No.~DP150101011).}
		\IEEEauthorblockA{Research School of Engineering,\\
			College of Engineering and Computer Science,\\
			The Australian National University,\\ Canberra, ACT 2601, Australia\\
			Email: rodney.kennedy@anu.edu.au} \and
		\IEEEauthorblockN{Aiden W.~Kennedy}
		\IEEEauthorblockA{Research School of Engineering,\\
			College of Engineering and Computer Science,\\
			The Australian National University,\\ Canberra, ACT 2601, Australia\\
			Email: collaborator@anu.edu.au}}
\else
	\author{Rodney~A.~Kennedy,~\IEEEmembership{Fellow,~IEEE}, and
		Collaborator,~\IEEEmembership{Senior Member,~IEEE}
		\thanks{Rodney~A.~Kennedy and Collaborator are with the Research School of Engineering,
			College of Engineering and Computer Science,
			The Australian National University (ANU), Canberra, ACT 2601, Australia
			(email: $\{$rodney.kennedy, collaborator$\}$@anu.edu.au).}
		\thanks{This work was supported under the Australian Research Council's Discovery Projects
			funding scheme (Project No.~DP1094350).}}
\fi


\maketitle

\newcommand{\R}{\mathscr{R}}
\newcommand{\taper}{w}%{w_{\R}^{\vphantom{g}}}

%\newcommand{\chfn}[2][]{\ifblank{#1}{\chi_{#2}}{\prescript{#1}{}\chi_{#2}}}
\newcommand{\chfn}{\chi_{\R}^{\vphantom{g}}}

\begin{abstract}
These are notes from scattered projects, held in one place a bit like a notebook.
\end{abstract}

\smallskip
\begin{IEEEkeywords}
key, word, keyword list
\end{IEEEkeywords}

\IEEEpeerreviewmaketitle

%\tableofcontents

%\clearpage

\section{Single Taper Design}


Classes of band-limited tapers for a spherical cap region are developed in \cite{Wieczorek:2005}.

\subsection{Constraints}

Here we seek a single taper $\taper(\unit{x})$ to use as a spatial-multiplicative window for a region $\R\subset\untsph$ on the sphere.  The sought properties are:
\begin{enumerate}
%\item unit energy on the sphere
%\[
%\int_{\untsph} \abs{\taper(\unit{x})}^2\,ds(\unit{x}) = 1
%\]
\item band-limited to degree $L$, that is,
\[
\coeff[\big]{\taper}_{\ell}^{m} = \innerp{\taper}{Y_{\ell}^{m}} = 0,\quad \ell>L,
\]
\item high spatial energy concentration, no less than some threshold $\lambda\in(0,1)$, that is,
\[
\frac{\displaystyle\int_{\R} \abs{\taper(\unit{x})}^2\,ds(\unit{x})}
{\displaystyle\int_{\untsph} \abs{\taper(\unit{x})}^2\,ds(\unit{x})} =
\frac{\norm[\big]{\taper}_{\R}^2}
{\norm[\big]{\taper}^2}
\geq \lambda
\]
and such functions are referred to as $\lambda$-concentrated

\item close to unity in the region of interest, that is, in some sense close to the characteristic function of the region $\R$
\[
\chfn(\unit{x}) =
\begin{cases}
1 & \unit{x}\in\R \\
0 & \text{otherwise}%\unit{x}\in\untsph\setminus\R
\end{cases}
\]
\end{enumerate}

\subsection{Caveats}

\begin{itemize}
\item
The threshold $\lambda\in(0,1)$ should not exceed the maximum theoretical spatial concentration. 

\item
The characteristic function is not band-limited.
\end{itemize}


\subsection{Use of taper}

When performing analysis a signal $f(\unit{x})$ localized to a region $\R$ we propose to use the modified signal
\[
\taper(\unit{x})\,f(\unit{x})
\]
which tends to concentrates the signal simultaneously in the spatial and spectral domains.

\subsection{Band-limited Slepian functions}

The band-limited Slepian functions for region $\R$ are denoted
$\varphi_n(\unit{x})$
with associated real positive eigenvalues $\lambda_n$, $n=1,2,\dotsc$, that is,
\[
\int_{\R} \abs{\varphi_n(\unit{x})}^2\,ds(\unit{x}) = \lambda_n.
\]
The Slepian functions are ordered in $n$ such that
\[
\lambda_n\geq\lambda_{n+1},\quad \forall n.
\]
Further the Slepian functions are orthonormal on $\untsph$ and orthogonal on $\R$.


\newcommand{\band}{s}
\newcommand{\bandt}{\band^{\lambda}}

\subsection{Formulation}

If $\band(\unit{x})$ is an $L$-band-limited function then by the completeness of the band-limited Slepian functions we have
\begin{equation}
\label{eqn:expan}
\band(\unit{x}) =
\sum_{n=1}^\infty \coeff[\big]{\band}_n\,\varphi_n(\unit{x}),
\end{equation}
in the sense of convergence in the mean, where
\begin{equation}
\label{eqn:coeff}
\coeff[\big]{\band}_n =
\innerp{\band}{\varphi_n} =
\int_{\untsph} \band(\unit{x})\,\conj{\varphi_n(\unit{x})}\,ds(\unit{x})
\end{equation}

Define $N(\lambda)$ such that
\begin{equation}
\label{eqn:lambda}
\lambda_{n}\geq \lambda \iff n\leq N(\lambda).
\end{equation}
Then the truncation
\begin{equation}
\label{eqn:expan2}
\bandt(\unit{x}) =
\sum_{n=1}^{N(\lambda)} \coeff[\big]{\bandt}_n\,\varphi_n(\unit{x})
\end{equation}
is $\lambda$-concentrated because we only use the $\lambda$-concentrated $L$-band-limited Slepian functions:
\[
\set[\Big]{\varphi_n(\unit{x})\colon
\int_{\R}\abs[\big]{\varphi_n(\unit{x})}^2\,ds(\unit{x}) = \lambda_n \geq \lambda}{\vphantom{\sum}}_{n=1}^{N(\lambda)}.
\]
That is, by the orthogonality of the Slepian functions on $\R$,
\begin{align*}
\norm[\big]{\bandt}_{\R}^2 &= \int_{\R} \abs[\big]{\bandt(\unit{x})}^{2}\,ds(\unit{x}) \\
&= \sum_{n=1}^{N(\lambda)} \abs[\big]{\coeff[\big]{\bandt}_n}^{2}
\int_{\R}\abs[\big]{\varphi_n(\unit{x})}^2\,ds(\unit{x}) \\
&= \sum_{n=1}^{N(\lambda)} \lambda_{n}\,\abs[\big]{\coeff[\big]{\bandt}_n}^{2}
\geq \lambda\sum_{n=1}^{N(\lambda)}\abs[\big]{\coeff[\big]{\bandt}_n}^{2} = \lambda\, \norm[\big]{\bandt}^2.
\end{align*}

\subsection{Window Design}

Functions satisfying expansion \eqref{eqn:expan2} form a finite $N(\lambda)$-dimensional space of $\lambda$-concentrated signals with the $\lambda$-concentrated $L$-band-limite Slepian functions as their basis.

Now the objective is to find a suitable $\band(\unit{x})$ that does not ``down-weight and ultimately discard'' the signal in the portions of the region $\R$ \cite{Wieczorek:2005}.  A function that equally weights all parts of the region is $\chi_{\R}^{\vphantom{g}}(\unit{x})$, the characteristic function of the region $\R$.  To obtain the optimal minimum mean square error between the inadmissible $\chi_{\R}^{\vphantom{g}}(\unit{x})$ and the finite $N(\lambda)$-dimensional subspace is through an orthogonal projection,
\begin{equation}
\taper(\unit{x}) = \sum_{n=1}^{N(\lambda)} \coeff[\big]{\taper}_n\,\varphi_n(\unit{x})
\end{equation}
where
\begin{align}
\coeff[\big]{\taper}_n = \innerp{\chi_{\R}^{\vphantom{g}}}{\varphi_n}
&= \int_{\untsph} \chi_{\R}^{\vphantom{g}}(\unit{x})\, \conj{\varphi_n(\unit{x})}\,ds(\unit{x}) \nonumber \\
&= \int_{\R} \conj{\varphi_n(\unit{x})}\,ds(\unit{x})
\end{align}

\bibliography{IEEEabrv,taper}

\end{document}


