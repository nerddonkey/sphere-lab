\documentclass[a4paper,10pt]{article}

\usepackage[margin=20mm]{geometry}


\usepackage{amsmath,amsthm,amssymb}
\usepackage{mathtools}
\usepackage{xcolor}

\newcommand{\dfn}{\triangleq}
 
\begin{document}

\section{Spherical Harmonics}

The most popular, call it standard, definition of the spherical harmonics is
\[
	Y_{\ell}^{m}(\theta,\varphi)\dfn
		N_{\ell}^{m}\,
		P_{\ell}^{m}(\cos\theta)\,
		e^{i m\varphi}
\]
where the normalization is given by
\[
	N_{\ell}^{m}\dfn\sqrt{\frac{2\ell+1}{4\pi}\frac{(\ell-m)!}{(\ell+m)!}}
\]

\section{Numerical considerations}

\subsection{Normalization}

As can be gleaned by looking at the asymptotics of the terms, computing $N_{\ell}^{m}$ and $P_{\ell}^{m}(z)$ \emph{separately} causes numerical problems as $\ell$ becomes large. Empirically, these problems emerge around $l=150$.

Direct computation of \emph{normalized or semi-normalized} associated Legendre functions, which essentially computes $N_{\ell}^{m}\,P_{\ell}^{m}(z)$ holistically, is a better numerical strategy.

We use the Schmidt semi-normalized Legendre functions because they are implemented in MATLAB. Empirically, if problems emerge then they are for $l>2000$.

\subsubsection{Schmidt semi-normalized associated Legendre functions}

As defined and implemented in MATLAB, the Schmidt semi-normalized (associated) Legendre functions are:
\color{blue}\begin{verbatim}
SP(N,M;X) = P(N,X), M = 0
          = (-1)^M * sqrt(2*(N-M)!/(N+M)!) * P(N,M;X), M > 0
\end{verbatim}\color{black}
that is,
\[
	S\!P_{\ell}^{m}(z)\dfn
		\begin{cases}
			\displaystyle
			P_{\ell}^{m}(z) &m=0 \\
			\displaystyle
			(-1)^{m}\sqrt{\frac{2(\ell-m)!}{(\ell+m)!}}\,
			P_{\ell}^{m}(z) &m>0\quad (m\leq\ell)
		\end{cases}
\]
In summary, the normalized associated Legendre functions and not the standard associated Legendre functions should be used when computing the spherical harmonics.

\subsection{Symmetry}

For negative $m$ we note
\[
	Y_{\ell}^{-m}(\theta,\varphi)=(-1)^{m}\,\overline{Y_{\ell}^{m}(\theta,\varphi)}
\]
so we only need to compute positive orders $m$.  That is, if we have computed $Y_{\ell}^{m}(\theta,\varphi)$ for $m>0$ then we can simply determine $Y_{\ell}^{m}(\theta,\varphi)$ for $m<0$.

In summary, with a few sign flips we can get the negative order spherical harmonics from the positive ones.

\subsection{More robust spherical harmonic computation}

Define
\[
	S_{\ell}^{m}(z)\dfn(-1)^{m}\sqrt{\frac{2(\ell-m)!}{(\ell+m)!}}\,
			P_{\ell}^{m}(z),\quad m\geq0\quad (m\leq\ell)
\]
and so we have
\begin{equation}
\label{eqn:sps}
	S_{\ell}^{m}(z)=
		\begin{cases}
			2\,S\!P_{\ell}^{0}(z) &m=0 \\[2mm]
			S\!P_{\ell}^{m}(z) &m>0
		\end{cases}
\end{equation}
in terms of the Schmidt semi-normalized Legendre functions. Further define
\[
	Q_{\ell}\dfn\sqrt{\frac{2\ell+1}{8\pi}}
\]
Then we have
\[
	Y_{\ell}^{m}(\theta,\varphi)=
		\begin{cases}
			\displaystyle
			(-1)^{m}\,Q_{\ell}\,
			S_{\ell}^{m}(\cos\theta)\,e^{im\varphi} &m\geq0\quad (0\leq m\leq\ell) \\[2mm]
			\displaystyle
			(-1)^{m}\,\overline{Y_{\ell}^{\smash{-m}}(\theta,\varphi)} &m<0\quad (-\ell\leq m<0)
		\end{cases}
\]

\subsubsection{MATLAB implementation}

The Schmidt semi-normalized Legendre functions are available in MATLAB so we only need to modify the $m=0$ case, \eqref{eqn:sps}.  The following code computes $Y_{\ell}^{m}(\theta,\varphi)$ for $\ell\geq0$ and $m\in\{-\ell,\dotsc,\ell\}$:
\color{blue}\begin{verbatim}
Sl=legendre(l,cos(theta),'sch');
Sl(1,:)=SPl(1,:)*sqrt(2); % m=0 adjustment
Ql=sqrt((2*l+1)/(8*pi));
Slm=Sl(abs(m)+1,:)'; % pull out S_l^|m| (evaluated at theta)
Ylm=(-1)^m*Ql*Slm*exp(1j*abs(m)*phi);
if m<0
   Ylm=(-1)^m*conj(Ylm);
end
\end{verbatim}\color{black}

\end{document}



